\documentclass{amsbook}
\usepackage{tikz}
\usepackage{todonotes}
\usepackage{amssymb,amsmath}
\usepackage{hyperref}

\newtheorem{definition}{Definition}

\title{
  The DAC programming language \\
  \large Distributed Algebraic Computations}

\begin{document}

\pagenumbering{Alph}
\begin{titlepage}
  \maketitle
  \thispagestyle{empty}
\end{titlepage}
\pagenumbering{arabic}

\tableofcontents
\thispagestyle{empty}
\newpage

\chapter{introduction}

\todo{Global optimaztion of algebraic structures...}
\todo{Create highlight rules for the listings latex package}

\begin{definition}
\cite{wiki:algebraicStructure} An \textbf{Algebraic structure} on a set $A$ (called carrier set or underlying set) is a collection of finitary operations on $A$; the set $A$ with this structure is also called an \textbf{algebra}.
\end{definition}

\chapter{Algebraic structures}
We take over the classification of algebraic structures from \cite{wiki:algebraicStructure}:

\begin{itemize}
\item Ring-like: Two binary operations, often called addition and multiplication, with multiplication distributing over addition. 
\item Group-like: One binary operation
\item Lattice-like: Two or more binary operations, including operations called meet and join, connected by the absorption law.
\item Module-like: Composite systems involving two sets and employing at least two binary operations.
\item Algebra-like: Composite system defined over two sets, a ring R and a R module M equipped with an operation called multiplication.
\end{itemize}

\begin{verbatim}
foo Algebra.Bar(param1, ...);
baz foo.Elem.Rand();
\end{verbatim}

\section{Ring-like}

\subsection{Machine native}
Analogously to typedef'ing, one can
\begin{verbatim}
foo Algebra.Ring.Int8;
baz foo.Elem = foo.Elem.Rand();
qux foo.Elem = 3;
corge := baz + qux;
\end{verbatim}

Or directly

\begin{verbatim}
baz Algebra.Ring.Int8.Elem = Algebra.Ring.Int8.Elem.Zero; // Additive Identity
qux Algebra.Ring.Int8.Elem = 2;
corge := baz + qux;
\end{verbatim}

For native types, it's possible to use shorthand.
\begin{verbatim}
baz int8 = int8.Rand();
qux int8 = int8.One; // Multiplicative Identity
corge := baz + qux;
\end{verbatim}

\begin{tabular}{l|l|l|l}
  \textbf{Ring} & \textbf{Min} & \textbf{Max} & \textbf{Comment}\\
  \hline
  int8 &  -128 & 127 & \\
  uint8 & 0 & 255 & \\
\end{tabular}

\begin{tabular}{l|l|l|l}
  \textbf{Field} & \textbf{Min} & \textbf{Max} & \textbf{Comment}\\
  \hline
  float32 & -Max & $(2-2^{-23})\times 2^{127}=3.4028235\times 10^{38}$ & 6\\
  cfloat32 & & & real/imaginary part are each float32  
\end{tabular}


\subsection{Non-machine-native}
These data types are usually non machine native and require additional computation steps.\todo{E.g. integer mod 42, or float32 mod 2, ...}

\begin{tabular}{l|l|l|l}
  \textbf{Ring} & \textbf{Min} & \textbf{Max} & \textbf{Comment}\\
  \hline
  int8\%n &  $-(n-1)$ & $n-1$ & for $n$ not prime. \\
\end{tabular}

\begin{tabular}{l|l|l|l}
  \textbf{Field} & \textbf{Min} & \textbf{Max} & \textbf{Comment}\\
  \hline
  int8\%n &  $-(n-1)$ & $n-1$ & for $n$ prime. \\
  Rational(int8) & & & Rational numbers over int8 \\
\end{tabular}

\subsection{Special Elements}
\begin{verbatim}
qux int8 = int8.One;
\end{verbatim}
\begin{tabular}{l|l}
  \textbf{Element} & \textbf{Comment} \\
  Zero & Additive Identity \\
  One & Multiplicative Identity \\
\end{tabular}

\section{Group-like}
\subsection{Matrix Groups\protect\footnote{\url{https://en.wikipedia.org/wiki/Linear_group}}}\hfill

\begin{tabular}{ll}
  \textbf{Action} & \textbf{Description} \\
  GL & General Linear \\
  SL & Special Linear \\
  O  & Orthogonal \\
  SO & Special Orthogonal \\
  U  & Unitary \\
  SU & Special Unitary \\
  Sp & Symplectic \\
  E  & Euclidean \\
  Lorentz & Lorentz, aka $O(1, n)$ or $O(n, 1)$ \\
  Poincare & Poincare \\
\end{tabular}


\section{Module-like}

\begin{tabular}{lll}
  \textbf{Algebra} & \textbf{Parameters} & \textbf{Description} \\
  Vector & data type, dimension & Vector space of specified dimension.\\
  Tensor & Product Factors & Vector space created by \\
  & prefix $\_~/~\hat{}$  denoting co/contravariance & the tensor product of given factors.\\
  Sum & Summands & Vector space created by \\
   & $\_~/~\hat{}$  denoting co/contravariance & the direct sum of given summands\\
\end{tabular}

\section{Lattice-like}

\begin{tabular}{lll}
  \textbf{Algebra} & \textbf{Parameters} & \textbf{Description} \\
  SimplicialComplex & vertices, faces, ... & Ordered by inclusion*
\end{tabular}

*See footnote\footnote{\url{https://en.wikipedia.org/w/index.php?title=Complete_lattice&oldid=867845390\#Examples}}

\subsection{Field}
rational Numbers

\begin{verbatim}
foo Algebra.Field.RationalNumbers(int8);
\end{verbatim}

\subsection{Functions}

\begin{verbatim}
foo Space.Vector(...);
func Add(v, w Space.Vector) v.Space {
     return v + w;
}
\end{verbatim}

\subsection{Actions}

\begin{verbatim}
foo Space.Bar(...);
baz foo.Action.Qux(...).Rand();
corge foo.Elem.Rand();
waldo := baz * corge;
\end{verbatim}


\subsubsection{Custom Actions}

\begin{verbatim}
foo Space.Bar(...);
baz Space.Qux(...);

func BarQuxAction(v Space.Bar, w Space.Qux) v.Space {
     ...
}

corge foo.Action.Custom(baz, BarQuxAction).Rand();
...
\end{verbatim}

\begin{thebibliography}{9}

\bibitem{wiki:algebraicStructure}
en.wikipedia.org,
\textit{Algebraic Structure},
\url{https://en.wikipedia.org/w/index.php?title=Algebraic_structure&oldid=884252836}
  
\end{thebibliography}

\end{document}
